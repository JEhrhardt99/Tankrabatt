\section[Determinants of Trade and Trade Costs]{Determinants of Trade and Trade Costs$^{\text{ MK}}$}
\label{sec:Determinants_of_Trade}

The initial gravity model as introduced by \textcite{tinbergen1962shaping} tried to explain and forecast the exports of one country to another by their economic masses and the geographical distance between them as the main proxies.
It is argued that deviations between the actual trade volumes and those estimated by the standard gravity equation as described may arise from factors acting as barriers to the exchange of goods and services or supporting bilateral trade among them.    

This section shows that these impacts promoting or hampering trade are directly related to trade costs and that they differ in their characteristics and the channel through which they exert their impacts. It is also presented how these trade costs and other determinants of trade are implemented into gravity equations in the empirical literature. 

\subsection[Trade Costs]{Trade Costs$^{\text{ MK}}$}
\label{sec:Proxies_Distance}

In order to understand what the term \textit{'trade costs'} describes, it might be helpful to have a look at a definition provided by the literature: \textit{\enquote{Broadly defined, trade costs include any cost of engaging in international trade such as transportation costs, tariffs, non-tariff barriers, informational costs, time costs and different product standards, among others.}} (\cite{CHEN2011206}). This underlines that trade costs do not only reflect costs in monetary terms but also factors that are not directly measurable or recognizable at first glance as a potential barrier or promoting aspect of trade but still have an impact on the decision of companies to trade with regard to the destination and the respective volume of goods and services to be exchanged.

As stated by \textcite{Anderson_2004} and \textcite{CHEN2011206}, researchers often face difficulties in finding a direct measure that also precisely captures all relevant trade costs. They trace this back mainly to constraints in data availability and the lack of empirical models for estimating trade costs that are not directly observable or where the required data is missing. Therefore, measures are constructed to approximate the impact of trade costs. For example, \textcite{CHEN2011206} derive a trade integration measure, or in the case of gravity equations, proxies are used to control for the influence of specific types of barriers or trade-promoting factors. Some of these proxies used in the empirical literature are presented in the next subsection \ref{sec:Proxies_Mass}.

In the literature, there are several approaches to classify trade costs. For example, \textcite{KHAN20111341} argue that trade costs can be distinguished by the place at which they occur. Therefore, they differentiate between \textit{'behind the border costs'} and \textit{'beyond the border costs'}. The first term describes all costs related to trade that arise before the arrival of the goods at their final destination. So, \textit{'behind the border costs'} comprise all factors that influence the initiation of contracts, such as national regulations in the home country, as well as other issues that affect the number of transactions and the volume of bilateral trade. All costs occurring in the country of destination are considered by \textcite{KHAN20111341} as \textit{'beyond the border costs'}. These are further subdivided into \textit{'explicit'} costs, including items such as tariffs and other directly measurable costs, and \textit{'implicit'} costs, comprising all factors that indirectly affect trade costs, such as regulations in the country of destination or the establishment of retail networks.

In this context, \textcite{KHAN20111341} and \textcite{Anderson_2004} argue that trade costs are not necessarily equal among partner countries but may vary based on specific characteristics and certain factors. They suggest that similarities between two countries, e.g. the same language and well-developed infrastructure, might lower trade costs, while trade regulations and restrictions, e.g. specific permission and certification requirements, can substantially increase the costs of trade. Additionally, they explain that political tensions between two countries might prevent companies from engaging in bilateral trade\footnote{\textcite{KHAN20111341} refer to the political tensions between India and Pakistan as an example where politics might affect trade relationships.}. 

\textcite{Anderson_2004}, \textcite{CHEN2011206} and \textcite{KHAN20111341} suggest that trade costs might vary between different goods. This is explained with specific transport and treatment requirements for certain goods, which raises the trade costs. Variations in the value-to-weight ratio among different goods are mentioned as a potentially important factor as well. A high value-to-weight ratio makes it more profitable to transport goods also across long distances\footnote{Goods with a low value-to-weight ratio either have both a low value and weight, or can be considered as heavy-weight goods with a comparably low value.}. Also, \textcite{CHEN2011206} argue that some barriers might have a more pronounced impact on specific goods, such as the effect of language on the trade of newspapers.

According to \textcite{Melitz_2003}, trade costs can be further distinguished as fixed and variable trade costs. Similar to the production costs, he suggests that some components might vary with the volume that is traded which are the variable costs of trade while some are independent of the quantities exchanged which are defined as the fixed costs of trade. \textcite{Melitz_2003} argues that these fixed components can be considered as market entry costs as they include all efforts and costs invested in the establishment of trade networks, research of relevant market information\footnote{\textcite{Melitz_2003} explains that new markets have to be explored before goods are exported. Exporters have to gather information about the regulatory requirements, laws and standards.} as well as marketing and retail expenses. All of these fixed cost components constitute sunk costs to the respective companies. The theoretical model introduced by \textcite{Melitz_2003} assumes that firms only participate in the market when all costs, fixed and variable costs, are covered and positive profits can be earned. 

In this context, the empirical analysis often distinguishes between the decision to engage in trade at all and the determination of the actual volumes that are traded. Here, the terms \textit{'extensive margin'} and the \textit{'intensive margin'} are important to mention. The first term is defined by \textcite{Lawless2010} as the number of companies engaging in export to a certain partner country, while the second term refers to the volumes that each company exports on average to the same destination\footnote{Not all researchers use the same definitions of the \textit{'extensive'} and \textit{'intensive'} margins, so that some variation can be found in the empirical literature.} It can expected that different trade cost components might exert a different impact on each margin in regards to their magnitude or direction.   

\subsection[Proxies for Trade Costs and Trade Determinants]{Proxies for Trade Costs and Trade Determinants$^{\text{ MK}}$}
\label{sec:Proxies_Mass}

The previous section has shown that trade costs can take different forms and are not always directly measurable in monetary terms. The gravity analysis includes trade cost components and other determinants of trade, mainly by proxies. In this section, these elements and their proxies are presented and it is explained through which channels they might affect the costs of trade. To provide a comprehensive overview, the components are grouped and assigned to different categories. 

\subsubsection[Economic and Policy Factors]{Economic and Policy Factors$^{\text{ MK}}$}
\label{EPF}

Economic, policy and institutional factors, in this context, consist of a wide range of variables that are incorporated to account for the impacts of economic mass, market size, trade policies as well as other regulations and institutional factors. 

The initial gravity equation by \textcite{tinbergen1962shaping} includes proxies for the economic mass of the home and partner country. In this case, the size of the economy was approximated by the gross national product (GNP). \textcite{tinbergen1962shaping} argues that larger economies are able to produce and trade more which in turn also raises their demand for foreign and domestic products. Other approaches use the gross domestic product (GDP) (see for example \textcite{Brau2013} or \textcite{Nordas2017}) or the GDP per capita (GDPPC) (incorporated, for instance, by \textcite{Kimura2006}) as an alternative proxy for economic mass. Market size is often described by the population of a country and included as a proxy in the gravity equation. For instance, \textcite{Kimura2006} include the size of the population in addition to the GDPPC\footnote{Note that trade volumes and GDPPC might be simultaneously determined, so the issue of endogeneity arises.}.   

Besides of economic mass, many approaches include additional economic or policy-related variables into the gravity equation, arguing that these might significantly affect trade costs. A proxy that is widely used, e.g. by \textcite{Kimura2006} and \textcite{DAS2023106246} among others, is a dummy variable that controls for the effect of free trade agreements (FTAs). It can be expected that FTAs reduce trade costs due to the reduction of tariffs and other trade-impeding measures. \textcite{Nordas2017} argue that it might also be reasonable to include a dummy variable for the memberships such as in the European Union (EU) which presents an even closer form of economic integration compared to a FTA. 

Additionally, currencies and exchange rates might affect the volume of bilateral trade. Therefore, some approaches, such as those of \textcite{Brau2013} or \textcite{SANTANAGALLEGO20161026}, control for the effect of a common currency, such as the Euro. Common currencies might reduce trade costs by lowering exchange rate uncertainty. To control for the effect of different currencies, for instance, \textcite{Tadesse2010} include the annual change of the exchange rate as proxy in the gravity equation. A fluctuation in the exchange rate can affect both imports and exports, while the concrete impact depends on the direction of the change in the exchange rate. For example, a devaluation usually promotes exports and reduces imports. 

For trade in services, \textcite{Nordas2017} argue that most barriers to trade might not always be directly observable and measurable due to their specific nature. They suggest that governments might impose specific restrictions and regulations on trade in services that not only affect the market entry of foreign competitors but also impose costs on domestic companies. Therefore, they conclude that legislative barriers must be closely examined and evaluated for their specific impact, which complicates the identification of such impediments. According to \textcite{Nordas2017}, these trade barriers could reduce imports by preventing the market entry of foreign competitors while exports could mainly be affected by low innovation incentives and a lack of efficiency as a consequence of a low number of other market participants. They suggest that especially the latter effect might also have implications on the supply and trade of goods that use services as an input the production process. \textcite{Nordas2017} propose to include the \textit{'Service Trade Restrictiveness Index' (STRI)} as introduced by the OECD as a proxy in the gravity equation to control for the impact of these policy restrictions on trade of services. As argued above, trade in goods might also be influenced by services trade restrictiveness. Therefore, incorporating the STRI in the gravity equation for trade in goods might capture this effect. \textcite{Kimura2006} use a similar approach by including the \textit{'Economic Freedom of the World' (EFW)} index as published by the Fraser Institute of Canada. In contrast to the \textit{STRI}, the \textit{EFW} considers trade barriers in the entire economy and is not limited to services trade and investments. The idea to control for the openness of an economy to trade is also adopted by \textcite{Brau2013}, using the total number of exports and imports as well as the number of destinations and countries of origins as a proxy. \textcite{Tadesse2010} include the sum of exports and imports in relation to the GDP of the respective country to approximate the openness of an economy. 

\subsubsection[Geographical and Transportation Factors]{Geographical and Transportation Factors$^{\text{ MK}}$}
\label{sec:geo}

Here, geographical factors account for all aspects related to the location of the country and its landscape features. Transportation factors, in this context, capture the impacts on all direct and indirect costs imposed by the shipment of goods. 

In accordance with the standard gravity equation by \textcite{tinbergen1962shaping}, transportation costs are usually approximated by the geographical distance between two countries. In this context, it is argued that a greater geographical distance is associated with higher shipping costs. Many studies, such as those by \textcite{Kimura2006} or \textcite{DAS2023106246}, use the geographical distance between the two capitals of the countries considered as a proxy in the gravity equation.  

Besides geographical distance, other factors might also affect the shipping costs of goods. A dummy variable controlling for a common border of two countries is widely used in the empirical literature, as seen in studies by \textcite{Kimura2006} or \textcite{DAS2023106246}, among others. A common border might reduce transportation costs and increase bilateral trade. One reason for this could be the facilitation of shipments regarding the means of transport, since potential unloading and reloading operations, for example in seaports, can be avoided. Other approaches, such as those by \textcite{SANTANAGALLEGO20161026}, include a dummy variable controlling for the effects of a country being surrounded only by sea (e.g. islands) and having no common land borders with its partner countries. It can be expected that this geographical feature increases the transportation costs due to specific transport requirements for the sea or air transport, such as packaging and labelling, and the need for multimodal transport solutions. In contrast, it can be argued that islands might not have the required capacities and capabilities to produce all goods and services on their own, which increases the need for imports and, consequently, promotes bilateral trade. Therefore, the effect of this proxy on trade volumes is expected to be more ambiguous. 

In general, it is expected that transportation costs exert a lower impact on the trade of services compared to the trade in goods due to the intangible nature of services. 

\subsubsection[Cultural Factors]{Cultural Factors$^{\text{ MK}}$}
\label{sec:culture}

Cultural factors, as defined here, encompass impacts on bilateral trade caused by differences in values, beliefs, communication and business practices among other things.

There is a variety of proxies used in gravity equations to control for cultural differences. One example included in the study by \textcite{SANTANAGALLEGO20161026}, is a dummy variable controlling for the effect of a common religion\footnote{A common religion, as defined by \textcite{SANTANAGALLEGO20161026}, has a share of more than 60 percent in the total population.}. Religious beliefs and traditions might affect business practices and the types of goods and services that are demanded and produced. Therefore, understanding these differences is crucial for companies engaging in foreign trade, but researching this information might increase the trade costs. 

Another proxy that is widely included in gravity equations, as for instance by \textcite{Kimura2006} or \textcite{Nordas2017} among others, controls for the impact of a common language by using a dummy variable. Speaking the same language can reduce communication costs and might also lower market entry barriers by facilitating the exchange of information.

Furthermore, gravity equations often control for the effects of a shared colonial history using a dummy variable. Such a proxy can be found, for example, in the analysis of \textcite{Nordas2017} or \textcite{DAS2023106246} among others. Historical linkages can contribute to reducing market entry barriers, as knowledge about foreign market conditions and cultural aspects may already exist. This could promote bilateral trade. 

\textcite{Tadesse2010} argue that cultural distance extends beyond those proxies introduced above and that measures of cultural distance should include more aspects of potential differences between societies. They define term 'culture' \textit{"(...) as an amalgam of its population's shared habits and traditions, learned beliefs and customs, attitudes, norms and values."} (\cite{Tadesse2010}). The authors suggest that these factors\footnote{For example, attitudes might be shaped not only by religious beliefs but also by personal experiences, socio-economic factors, or other historical developments than a common colonial background.}, not yet presented, also significantly impact business practices and institutions, among other things. \textcite{Tadesse2010} consider that cultural distance goes along with a lack of trust and information. Consequently, the trade costs and barriers should be higher for countries with greater cultural differences. Therefore, the authors construct a measure approximating the extent to which norms and attitudes diverge between two countries and incorporate it into a gravity framework. The measure is based on representative survey data covering ethical, policy, social and economic aspects.  

Due to the intangible nature of services and potentially higher requirements for communication and exchange of information, one could expect that cultural factors might affect the trade in services to a higher extent than the trade in goods. 

\subsubsection[Mobility]{Mobility$^{\text{ MK}}$}
\label{sec:mobility}

Mobility, as understood in this context, includes all types associated with the movement of people. The empirical literature mainly incorporates mobility by analysing the impacts of migration and tourism on bilateral trade. 

\textcite{Brau2013} argues that tourism has a twofold impact on bilateral trade. On one hand, it is regarded as an export of services and thereby a direct component of trade, while on the other hand, tourism might promote the exchange of goods between two countries by reducing trade costs. The latter is explained by \textcite{Brau2013} with the fact that tourism is connected with an exchange of information between locals and those visiting the host country. Therefore, tourists learn about cultural aspects, preferences, and the variety of goods available within the country. This could reduce fixed costs involved in gathering information about foreign markets. Furthermore, they argue that better knowledge about foreign products might create a demand for these commodities in the home country of the tourists. Referring to \textcite{Kulendran_2000}, they suggest that potential business opportunities are revealed to both the visitors and locals of the respective country, which might further promote bilateral trade between the two economies and lead to a further reduction of fixed costs. Similar arguments as those presented by \textcite{Brau2013} are also made by \textcite{SANTANAGALLEGO20161026}, who further argue that tourism might improve the infrastructural conditions necessary for the transportation of people and goods, such as airports, road conditions, and telecommunication networks, among other things, which can facilitate trade and reduce communication and transportation costs. Both authors include tourism as the number of tourist arrivals to the respective country whose bilateral trade flows are analysed. 

%\textcite{Brau2013} examine the effect of tourism on the exports of the respective country by means of a gravity model focusing on 25 European countries in the period between 1998 and 2009. Their analysis distinguishes between consumption and other goods\footnote{As per the definition used by \textcite{Brau2013}, other goods comprise primary, capital and intermediate goods.} and between different sectors as defined by the ISIC, Rev. 4 standard, as it is assumed that tourists get more likely in contact with products of specific types enabling a better exchange of information for these goods. The presented results indicate that tourism significantly promotes trade of final goods while negative effects, mainly non-significant, were found for the exports of other goods\footnote{The results are consistent across different model specifications adding different controls or controlling for previous and new member states of the European Union.}. The analysis of different consumption good sectors, provides mainly positive coefficients for the number of tourist arrivals, although not all of them were found to be significant.
%\textcite{SANTANAGALLEGO20161026} analyse the impact of tourism on the exports of the respective host country referring to the model introduced by \textcite{HelpmanElhanan2008ETFT} as an extension to the gravity model. A particular interest of their research is the effect of tourism on the extensive and the intensive margin of trade. Due to the underlying model, \textcite{SANTANAGALLEGO20161026} use other definitions of the two trade margins. The extensive margin describes the probability that a trade relationship is established between two countries and is estimated by a probit model. The intensive margin is defined as the total volume of exports. Their analysis is based on data from 195 countries for the year 2012. The estimated results suggest that tourism has a positive effect on both, the extensive and the intensive margin of trade.
The empirical literature also suggests a positive effect of migration on trade. \textcite{Head1998} argue that immigrants might have potential advantages in establishing trade relationships with their home countries due to their specific knowledge about market and cultural factors, understanding of the legislative framework, and possession of language skills and potential network connections. Thus, migration could potentially reduce fixed trade costs as well as communication costs. Furthermore, they propose that immigrants might measurably impact the volume of imports by expressing demand for products from their home country. Additionally, they argue that the extent to which immigration can impact the bilateral trade between the former and the new home country may vary with the causes that lead to migration. For example, \textcite{Head1998} expect that refugees do not have a significant impact on trade as they often leave their home countries due to wars, persecution or other factors that might prevent them to maintain linkages to their home countries. %In their gravity analysis, \textcite{Head1998} find a significant positive impact of migration on Canadian imports and exports, while the imports seem to be more affected. Their research considers data for Canada from 1980 to 1992. The effected of migration was incorporated into the gravity equation as the total number of immigrants from a certain country. Furthermore, \textcite{Head1998} find evidence that the effect on imports and exports differ with the reason of migration and the region from which people migrate to Canada.   

As described above, mobility factors mainly influence cultural and communication barriers and reduce the related trade costs. Although the studies mentioned here do not explicitly analyse trade in services, a significant impact on the trade volumes of the service sector is expected for the reasons mentioned earlier. However, it cannot be directly inferred which trade flow - goods or services - are affected to a greater extent. 

\subsubsection[Other factors]{Other factors$^{\text{ MK}}$}
\label{sec:other}

In this context, 'other factors' encompass all aspects that cannot be directly attributed to the aforementioned categories. Thus, the proxies presented are not necessarily related to each other but may still have a significant effect on bilateral trade and the associated trade costs. 

\textcite{Freund_2004} analyse the effect of the internet on exports between two countries. They argue that the internet reduces fixed costs facilitating the identification of potential suppliers and customers in foreign markets and by gathering information on market conditions. Additionally, they consider the internet to be a cheaper option for the exchange of information. Consequently, they argue that lower entry costs and barriers, may induce more exporters to enter the market, thereby increase the volume of exports. Regarding services trade, \textcite{Freund_2004} suggest that the internet increases the number of potential services that can be traded, for example, by establishing new business models and service products that can be traded via the internet. \textcite{Freund_2004} propose to include the effect of the internet in the gravity equation using the number of web hosts or the number of internet users per country. In contrast, \textcite{DAS2023106246} use the percentage share of internet users within both countries to control for the effect of the internet on bilateral trade\footnote{Note the time difference of the two studies.}.

A more recent paper by \textcite{Martinez2023} examines the impact of climate change on bilateral trade. They identify three possible channel through which climate change might impact trade flows between two countries. First, extreme weather events and natural catastrophes, such as floods or storms, as a direct or indirect consequences of climate change, could damage the infrastructure and increase maintenance requirements. Consequently, specific infrastructural facilities might not be usable for a certain period of time due reconstruction or maintenance. This could increase trade costs due to, for example, required changes of trade routes or possible delays with the processing and transport of goods. The second channel refers to the consequences of climate change on humans. \textcite{Martinez2023} argue that extreme temperatures might reduce the productivity of workers affecting the production of goods and services for domestic and international trade. Additionally, climate change could increase the probability of pandemics, such as the COVID-19 pandemic in 2020, and the transmission of diseases by insects, such as mosquitoes. Lastly, they suggest that transport costs could be reduced due to the melting of ice in the Arctic region, enabling cargo vessels to use shorter and possibly cheaper sea routes, potentially avoiding the Panama or Suez Canal. \textcite{Martinez2023} include climate change in the gravity equation by using the average annual temperature and the number of extreme weather events\footnote{\textcite{Martinez2023} include following types of events: "wildfires, floods, extreme temperatures, epidemics, insect infestation, storms, droughts and landslides" (\cite{Martinez2023})} as proxies.































